For decades scientists have attempted to use ideas of classical mechanics to choose basis functions
for calculating spectra. The hope is that a classically-motivated basis set will be small because it
covers only the dynamically important part of phase space. One popular idea is to use phase space 
localized (PSL) basis functions. This thesis improves on previous efforts to use PSL functions and 
examines the usefulness of these improvements. Because  the overlap matrix, in the matrix eigenvalue 
problem obtained by using PSL functions with the variational method, is not an identity, it is costly 
to use iterative methods
to solve the matrix eigenvalue problem. We show that it is possible to circumvent
the orthogonality (overlap) problem and use iterative eigensolvers. We also present an altered method of calculating the matrix elements that
improves the performance of the PSL basis functions, and also a new method which more efficiently chooses which
PSL functions to include. These improvements are applied to a variety of single well molecules. We conclude that for single minimum molecules, the PSL functions are inferior to other basis functions.  However, the ideas developed here can be applied to other types of basis functions, and PSL functions may be useful for multi-well systems. 